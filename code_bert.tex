\definecolor{codegreen}{rgb}{0,0.6,0}
\definecolor{codegray}{rgb}{0.5,0.5,0.5}
\definecolor{codepurple}{rgb}{0.58,0,0.82}
\definecolor{backcolour}{rgb}{0.95,0.95,0.92}
 
\lstdefinestyle{mystyle}{
    backgroundcolor=\color{backcolour},   
    commentstyle=\color{codegreen},
    keywordstyle=\color{magenta},
    numberstyle=\tiny\color{codegray},
    stringstyle=\color{codepurple},
    basicstyle=\ttfamily\footnotesize,
    breakatwhitespace=false,         
    breaklines=true,                 
    captionpos=t,                    
    keepspaces=true,                 
    numbers=left,                    
    numbersep=5pt,                  
    showspaces=false,                
    showstringspaces=false,
    showtabs=false,                  
    tabsize=2
}
 
\lstset{style=mystyle}

\begin{lstlisting}[language=Python, caption=My attempt to adapt BERT to short text classification]
from __future__ import absolute_import, division, print_function, unicode_literals

import os
import re
import pandas as pd
import numpy as np
import tensorflow as tf
import tensorflow_hub as hub
from tensorflow import keras
from sklearn import preprocessing  # for labelencoding
from sklearn.model_selection import train_test_split
from datetime import datetime
import bert
from bert import run_classifier
from bert import optimization
from bert import tokenization

# Initialize session
sess = tf.Session()

print("done importing packages")
print("Version: ", tf.__version__)
print("GPU is", "available" if tf.config.experimental.list_physical_devices("GPU") else "NOT AVAILABLE")

DATA_DIR = "data/"
training_data = format(DATA_DIR + "train.csv")
validation_data = format(DATA_DIR + "val.csv")

print(training_data)
print(validation_data)

DATA_LIMIT = 0  # set to 0 for no limits, useful for debugging though it is not representative of actual performance

# DATA Cleaning
print("Data Cleaning")

# Applying cleaning of data due to the same suspicious as Jeffrey paper suggest
print("# Applying cleaning of data due to the same suspicious as Jeffrey paper suggest")
print("def cleanSentence")


def cleanSentence(text):
    return re.sub("[^a-zA-Z0-9-]", " ", text)


print("start def suspicious_whitespace_remover")


def suspicious_whitespace_remover(text):
    return re.sub(r"\s{2,}", " ", text).strip()


# DATA Preparation
print("Data preparation")

# prepare data: we just need this to look up the feature vector of every word we can encounter.
# unsure whether to keep this in a function
print("def preparedataset")


def preparedataset(data_path, data_limit=DATA_LIMIT, shuffle=True, reverse=False):
    print(data_path)
    X, y = [], []
    i = 0
    with open(data_path, "r") as infile:
        for line in infile:
            a = line.split("|")
            if len(a) != 3:
                print(len(a))
                print(a)
            label, text, bs = line.split("|")
            # texts are already cleaned, just split on |. BS includes the time stamp (not used at this time)
            # a way to get around this seeing the csv headers as actual data entries. We don't want any of that
            if i > 0:
                label = "0" + label if len(label) < 8 else label  # hacky way to solve yet another string to int issue
                if reverse:
                    print(text)
                    print(reverseSentence(text))
                    X.append(reverseSentence(text))
                    y.append(label[:2])

                # cleanup - perhaps make this faster?
                text = cleanSentence(text)
                text = suspicious_whitespace_remover(text)

                X.append(text)
                y.append(label[:2])

            i = i + 1  # temporary solution pending pipeline changes

    X, y = np.array(X), np.array(y)

    # shuffle the data
    if shuffle:
        idx = np.random.permutation(len(y))
        X, y = X[idx], y[idx]

    # apply limit
    if data_limit > 0:
        X, y = X[:data_limit], y[:data_limit]

    return X, y


print("start preparing data set")

texts, y = preparedataset(training_data)
print("total examples: %s" % len(y))
print(type(texts[0]))
print(texts[0])

print("finished preparing data set")

print("starting preprocess label")

le = preprocessing.LabelEncoder()
le.fit(y)
labels = le.transform(y)
len(le.classes_)

print(type(y[0]))
print(type(le.classes_))

train_df = texts
train_label = tf.keras.utils.to_categorical(np.asarray(labels))
print('Shape of label tensor:', labels.shape)

print("get the validation data")

X_v, y_val = preparedataset(validation_data, data_limit=0, reverse=False)
np.shape(y_val)

print("end preprocess files")

# Data Preprocessing
print("Data Preprocessing")

# train_df = texts
# test_df = validation_data

# print(type(texts[0]))
# print(type(train_df[0]))


class InputExample(object):
    """A single training/test example for simple sequence classification."""

    def __init__(self, guid, text_a, text_b=None, labels=None):
        """Constructs a InputExample.

        Args:
            guid: Unique id for the example.
            text_a: string. The untokenized text of the first sequence. For single
            sequence tasks, only this sequence must be specified.
            text_b: (Optional) string. The untokenized text of the second sequence.
            Only must be specified for sequence pair tasks.
            labels: (Optional) [string]. The label of the example. This should be
            specified for train and dev examples, but not for test examples.
        """
        self.guid = guid
        self.text_a = text_a
        self.text_b = text_b
        self.labels = labels


print("input example done")


class InputFeatures(object):
    """A single set of features of data."""

    def __init__(self, input_ids, input_mask, segment_ids, label_ids, is_real_example=True):
        self.input_ids = input_ids
        self.input_mask = input_mask
        self.segment_ids = segment_ids
        self.label_ids = label_ids,
        self.is_real_example = is_real_example


print("input features done")

print("start defining tokenizer")


def create_tokenizer_from_hub_module(bert_path):
    """Get the vocab file and casing info from the Hub module."""
    bert_module = hub.Module(bert_path)
    tokenization_info = bert_module(signature="tokenization_info", as_dict=True)
    vocab_file, do_lower_case = sess.run(
        [tokenization_info["vocab_file"], tokenization_info["do_lower_case"]]
    )

    return FullTokenizer(vocab_file=vocab_file, do_lower_case=do_lower_case)


print("defining tokenizer end")

print("convert single example")


def convert_single_example(tokenizer, example, max_seq_length=256):
    """Converts a single `InputExample` into a single `InputFeatures`."""

    if isinstance(example, PaddingInputExample):
        input_ids = [0] * max_seq_length
        input_mask = [0] * max_seq_length
        segment_ids = [0] * max_seq_length
        label = 0
        return input_ids, input_mask, segment_ids, label

    tokens_a = tokenizer.tokenize(example.text_a)
    if len(tokens_a) > max_seq_length - 2:
        tokens_a = tokens_a[0: (max_seq_length - 2)]

    tokens = []
    segment_ids = []
    tokens.append("[CLS]")
    segment_ids.append(0)
    for token in tokens_a:
        tokens.append(token)
        segment_ids.append(0)
    tokens.append("[SEP]")
    segment_ids.append(0)

    input_ids = tokenizer.convert_tokens_to_ids(tokens)

    # The mask has 1 for real tokens and 0 for padding tokens. Only real
    # tokens are attended to.
    input_mask = [1] * len(input_ids)

    # Zero-pad up to the sequence length.
    while len(input_ids) < max_seq_length:
        input_ids.append(0)
        input_mask.append(0)
        segment_ids.append(0)

    assert len(input_ids) == max_seq_length
    assert len(input_mask) == max_seq_length
    assert len(segment_ids) == max_seq_length

    return input_ids, input_mask, segment_ids, example.label


print("convert examples to features")


def convert_examples_to_features(tokenizer, examples, max_seq_length=256):
    """Convert a set of `InputExample`s to a list of `InputFeatures`."""

    input_ids, input_masks, segment_ids, labels = [], [], [], []
    for example in tqdm(examples, desc="Converting examples to features"):
        input_id, input_mask, segment_id, label = convert_single_example(
            tokenizer, example, max_seq_length
        )
        input_ids.append(input_id)
        input_masks.append(input_mask)
        segment_ids.append(segment_id)
        labels.append(label)
    return (
        np.array(input_ids),
        np.array(input_masks),
        np.array(segment_ids),
        np.array(labels).reshape(-1, 1),
    )


print("convert_text_to_examples")


def convert_text_to_examples(texts, labels):
    """Create InputExamples"""
    InputExamples = []
    for text, label in zip(texts, labels):
        InputExamples.append(
            InputExample(guid=None, text_a=" ".join(text), text_b=None, label=labels)
        )
    return InputExamples


print("Bert Layer")


class BertLayer(tf.keras.layers.Layer):
    def __init__(
            self,
            n_fine_tune_layers=10,
            pooling="mean",
            bert_path="https://tfhub.dev/google/bert_multi_cased_L-12_H-768_A-12/1",
            **kwargs,
    ):
        self.n_fine_tune_layers = n_fine_tune_layers
        self.trainable = True
        self.output_size = 768
        self.pooling = pooling
        self.bert_path = bert_path
        if self.pooling not in ["first", "mean"]:
            raise NameError(
                f"Undefined pooling type (must be either first or mean, but is {self.pooling}"
            )

        super(BertLayer, self).__init__(**kwargs)
        print("layer > build")

    def build(self, input_shape):
        self.bert = hub.Module(
            self.bert_path, trainable=self.trainable, name=f"{self.name}_module"
        )

        # Remove unused layers
        trainable_vars = self.bert.variables
        if self.pooling == "first":
            trainable_vars = [var for var in trainable_vars if not "/cls/" in var.name]
            trainable_layers = ["pooler/dense"]

        elif self.pooling == "mean":
            trainable_vars = [
                var
                for var in trainable_vars
                if not "/cls/" in var.name and not "/pooler/" in var.name
            ]
            trainable_layers = []
        else:
            raise NameError(
                f"Undefined pooling type (must be either first or mean, but is {self.pooling}"
            )

        # Select how many layers to fine tune
        for i in range(self.n_fine_tune_layers):
            trainable_layers.append(f"encoder/layer_{str(11 - i)}")

        # Update trainable vars to contain only the specified layers
        trainable_vars = [
            var
            for var in trainable_vars
            if any([l in var.name for l in trainable_layers])
        ]

        # Add to trainable weights
        for var in trainable_vars:
            self._trainable_weights.append(var)

        for var in self.bert.variables:
            if var not in self._trainable_weights:
                self._non_trainable_weights.append(var)

        super(BertLayer, self).build(input_shape)
        print("layer > call")

    def call(self, inputs):
        inputs = [K.cast(x, dtype="int32") for x in inputs]
        input_ids, input_mask, segment_ids = inputs
        bert_inputs = dict(
            input_ids=input_ids, input_mask=input_mask, segment_ids=segment_ids
        )
        if self.pooling == "first":
            pooled = self.bert(inputs=bert_inputs, signature="tokens", as_dict=True)[
                "pooled_output"
            ]
        elif self.pooling == "mean":
            result = self.bert(inputs=bert_inputs, signature="tokens", as_dict=True)[
                "sequence_output"
            ]

            mul_mask = lambda x, m: x * tf.expand_dims(m, axis=-1)
            masked_reduce_mean = lambda x, m: tf.reduce_sum(mul_mask(x, m), axis=1) / (
                    tf.reduce_sum(m, axis=1, keepdims=True) + 1e-10)
            input_mask = tf.cast(input_mask, tf.float32)
            pooled = masked_reduce_mean(result, input_mask)
        else:
            raise NameError(f"Undefined pooling type (must be either first or mean, but is {self.pooling}")

        return pooled

    def compute_output_shape(self, input_shape):
        print("layer > compute output shape")
        return (input_shape[0], self.output_size)


# Build model
print("Build model")


def build_model(max_seq_length):
    in_id = tf.keras.layers.Input(shape=(max_seq_length,), name="input_ids")
    in_mask = tf.keras.layers.Input(shape=(max_seq_length,), name="input_masks")
    in_segment = tf.keras.layers.Input(shape=(max_seq_length,), name="segment_ids")
    bert_inputs = [in_id, in_mask, in_segment]

    bert_output = BertLayer(n_fine_tune_layers=3)(bert_inputs)
    dense = tf.keras.layers.Dense(256, activation="relu")(bert_output)
    pred = tf.keras.layers.Dense(len(le.classes_), activation="softmax")(dense)

    model = tf.keras.models.Model(inputs=bert_inputs, outputs=pred)
    model.compile(loss="categorical_crossentropy", optimizer="adam", metrics=["accuracy"])
    model.summary()

    return model


print("init vars")


def initialize_vars(sess):
    sess.run(tf.local_variables_initializer())
    sess.run(tf.global_variables_initializer())
    sess.run(tf.tables_initializer())
    K.set_session(sess)


print("main")


def main():
    # Params for bert model and tokenization
    bert_path = "https://tfhub.dev/google/bert_multi_cased_L-12_H-768_A-12/1"
    max_seq_length = 256

    # Create datasets (Only take up to max_seq_length words for memory)
    train_text = texts.tolist()
    # train_text = [" ".join(t.split()[0:max_seq_length]) for t in train_text]
    train_text = np.array(train_text, dtype=object)[:, np.newaxis]
    train_label = labels.tolist()

    test_text = texts.tolist()
    # test_text = [" ".join(t.split()[0:max_seq_length]) for t in test_text]
    test_text = np.array(test_text, dtype=object)[:, np.newaxis]
    test_label = labels.tolist()

    # Instantiate tokenizer
    tokenizer = create_tokenizer_from_hub_module(bert_path)

    # Convert data to InputExample format
    train_examples = convert_text_to_examples(train_text, train_label)
    test_examples = convert_text_to_examples(test_text, test_label)

    # Convert to features
    (
        train_input_ids,
        train_input_masks,
        train_segment_ids,
        train_labels,
    ) = convert_examples_to_features(
        tokenizer, train_examples, max_seq_length=max_seq_length
    )
    (
        test_input_ids,
        test_input_masks,
        test_segment_ids,
        test_labels,
    ) = convert_examples_to_features(
        tokenizer, test_examples, max_seq_length=max_seq_length
    )

    model = build_model(max_seq_length)

    # Instantiate variables
    initialize_vars(sess)

    model.fit(
        [train_input_ids, train_input_masks, train_segment_ids],
        train_labels,
        validation_data=(
            [test_input_ids, test_input_masks, test_segment_ids],
            test_labels,
        ),
        epochs=5,
        batch_size=32,
    )


if __name__ == "__main__":
    main()

\end{lstlisting}
